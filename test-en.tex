\documentclass[16pt, oneside]{report}
\usepackage[francais]{babel}
\usepackage{textcomp}
 \usepackage[pdftex]{graphicx}
 \usepackage{grffile}  
\font\myfont=cmr12 at 40pt
\graphicspath{{./images/}{./}}  


\title{{\myfont Open Project}}
\author{mbourlet, gfernand, fbuoro, mchoong \#e1r6}
\date{}

\begin{document}

\maketitle

\chapter{Project presentation}
The goal of our project is to achieve a native video game on Android. \\
This game will be a battle RPG turn by turn but with a profound strategical aspect , the player can not choose his actions at each turn, instead he will have to set a basic IA (see below). \\
The project will be split into two parts: the engine and the game.

\section{Engine}
The engine will be a basic engine for the game, it will have to be reusable on different projects . Its purpose will be to accelerate the future and current development of the game. It is divided into several Manager.

\subsection{Game}
Game class is the central class of the engine. He's the one who manage the program logic with his State System. \\
At launch the class tries to automatically load the save.

\subsection{Le DataManager}
The DataManager is the one who manages the backups of the user. It offers two possibilities :
\begin {itemize}
	\item Local Save : The file is stored on the user's phone in the application files .
	\item Cloud Save : If the user logs in with a Google+ account and gives us the authorization, the DataManager also backup to the cloud. \\ 
\end {itemize}


\subsection{State}
The states are the different states of the game, they will manage the navigation in the application.\\

\section{Projet}
The project gonna use our engine. He'll have to use all the technologies offered by Google, such as cloud storage , multiplayer , achievements  etc.

\chapter{IA}
The player will be able to control a team of 3 characters . \\
Each character will be able to have many classes ( Warrior , Mage, Healer ) \\
Each class of each character will have its own level. \\
When a character levels up in a class , he gains tools to program the AI of this class, these tools can be decompose into several types :
\begin {itemize}
\item Condition: The behaviour of the AI
\item Action: This can be spells, attacks, etc ...
\item Slot : To ask a new set of conditions / actions
\end {itemize}

\section{Classe ?}
Each character has multiple classes , the player can decide to change the classes of these characters during the fight to vary the strategies according to the opponent.\\\\
Depending on the use frequency of a class in a fight , this one earns more or less percentage of the amount of experience given by this fight.

\chapter{Planning de r\'ealisation}
The project will use the Agile method (Sprint + meeting of 15 minutes every day to make a quick assessment ) . \\
However, we gonna achieve many iterations for the project. Each iteration 'll aim to make a functional product , consisting of an improvement of the previous one.
\begin{itemize}
	\item Iteration 1: Realiser un combat entre deux IA basique (Allie (programmable) / Enemie)
	\item Iteration 2: Realiser un systeme basique de classe 
	\item Iteration 3: Realiser un systeme de niveau + inventaire
	\item Iteration 4: Realiser un systeme d'histoire / quete basique
	\item Iteration 5: Ajout d'IA enemie
	\item Iteration 6: Amelioration du systeme d'histoire / quete (branche)
\end{itemize}
A la fin de ces iterations, nous pourrons considerer que la partie obligatoire est finies. Nous pr�voyons au cas ou de r�aliser quelque bonus plus ou moins dur, pouvant �tre r�aliser en plus:
\begin{itemize}
	\item Ajout des achievements
	\item Beaucoup d'item
	\item Plus de classes
	\item Ladder solo
	\item Multijoueur
	\item Ladder multijoueur
	\item Animation entiere du personnage
\end{itemize}

\chapter{Design}

Le jeu sera r\'ealis\'e en pixel art. Chaque partie du corps des personnages et chaque item seront en r�alit� diff\'erents calques que nous superposeront pour cr\'eer les images finales. \\
Cela nous permets de r\'ealiser rapidement un grand nombre de sets de personnages et de rendre nos personnages principaux facilement personnalisables. 
Voici un exemple de nos premi\`eres r\'ealisations:\\
\\
\\
\includegraphics[width=0.3\textwidth]{character-base}
\includegraphics[width=0.3\textwidth]{character-head}
\includegraphics[width=0.3\textwidth]{character-robe}
\includegraphics[width=0.3\textwidth]{character-wand}
\\
\\
\\
\begin{center}
\includegraphics[width=0.3\textwidth]{character}
\end{center}

\chapter{Communication}
La communication se feras via un Website/devblog et sera appui\'ee par un compte twitter qui d\'etaillera l'avancement du jeu en postant des photos, vid\'dos de gameplay etc. Le but \'etant d'attirer le plus de personnes \`a s'int�resser au projet avant m\^eme la sortie du jeu. C'est notamment pour cela qu'une phase de beta test sera disponible aux utilisateurs les plus interess\'ess\'es.\\
 La beta test sera g\'er\'ee a l'aide de la google d�velopper console. Le site proposera \'egalement l'inscription \`a une newsletter. 

\chapter{Mon\'etisation}
La mon\'etisation du jeu se fera via les achats in-app et les pubs. 
Les achats in-app ne devront pas bousculer l'\'equilibre du jeu, notamment dans l'optique d'un multijoueur le plus strat\'egique possible. Nous pr\'evoyons pour le moment l'achat de packs d'xp qui permettrait de lvl up plus vite. \\
Les pubs se voudront non-intrusive pour ne pas g\^acher l'\'experience utilisateur. Elles pourront \^etre par exemple integr\'ees sous forme de banni�re a l'\'ecran de score d'un combat. Elles ne devront pas g\^ener la navigation. Nous utiliserons des api comme admob (utilis\'e par Rovio, Backflip Studio, Fingersoft... )pour int\'egrer des pubs cibl\'ees et donc augmenter nos revenus.

\chapter{Elements \`a v\'erifier en fin de projet}
\begin{itemize}
	\item Un syst\`eme d'IA fonctionnel (voir chapitre Syst\`eme d'IA)
	\item Une interface agr\'eable
	\item Un mode solo fonctionnel
	\item Un multijoueur fonctionnel
	\item La pr\'esence d'achievements
\end{itemize}

\chapter{Bonus}
\begin{itemize}
	\item Une mon\'etisation bien int\'egr\'ee
\end{itemize}

\end{document}
