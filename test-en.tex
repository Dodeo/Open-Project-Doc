\documentclass[16pt, oneside]{report}
\usepackage[francais]{babel}
\usepackage{textcomp}
 \usepackage[pdftex]{graphicx}
 \usepackage{grffile}  
\font\myfont=cmr12 at 40pt
\graphicspath{{./images/}{./}}  


\title{{\myfont Open Project}}
\author{mbourlet, gfernand, fbuoro, mchoong \#e1r6}
\date{}

\begin{document}

\maketitle

\chapter{Project presentation}
The goal of our project is to achieve a native video game on Android. \\
This game will be a battle RPG turn by turn but with a profound strategical aspect , the player can not choose his actions at each turn, instead he will have to set a basic IA (see below). \\
The project will be split into two parts: the engine and the game.

\section{Engine}
The engine will be a basic engine for the game, it will have to be reusable on different projects . Its purpose will be to accelerate the future and current development of the game. It is divided into several Manager.

\subsection{Game}
Game class is the central class of the engine. He's the one who manage the program logic with his State System. \\
At launch the class tries to automatically connect to the google api for user connects and takes advantage of the achievements, the backup to the cloud etc. \\
The user may refuse the connection. If he refuses the connection three times at launch, the program will no longer launch the automatic connection. This is in stock in a given cache file . \\
Before starting the game, the game class will attempt to load a backup via the DataManager .

\subsection{Le DataManager}
The DataManager is the one who manages the backups of the user. It offers two possibilities :
\begin {itemize}
	\item Local Save : The file is stored on the user's phone in the application files .
	\item Cloud Save : If the user logs in with a Google+ account and gives us the authorization, the DataManager also backup to the cloud. \\ 
\end {itemize}
 The work of the DataManager is to facilitate the backup spots and loading of data . It choose the correct backup according to a timestamp adding to the file and by managing conflicts if necessary. When saving he save either locally or in locally + cloud connection according to the Google API connection state.


\subsection{State}
The states are the different states of the game, they will manage the navigation in the application.\\
MENGWEI HELP ME TO EXPLAIN STATE :'(


\section{Projet}
The project gonna use our engine. He'll have to use all the technologies offered by Google, such as cloud storage , multiplayer , achievements  etc.

\chapter{IA}
Le joueur pourras contr\^oler une \'equipe de 3 personnages.\\
Chaque personnage pourras avoir de nombreuses classes (Guerrier, Mage, Soigneur)\\
Chaque classe de chaque personnage auras son propre niveau.\\
Quand un personnage augmente de niveau dans une classe, il gagne des outils pour programmer son IA li\'e a cette classe, ces outils peuvent \^etre d\'ecomposer en plusieurs types:
\begin{itemize}
	\item Condition: Permet de conditionner le comportement de l\textquotesingle IA
	\item Action: Cela peut \^etre des sorts, des attaques, etc...
	\item Slot: Permet de poser une nouvelle paire de conditions/actions
\end{itemize}

\section{Classe ?}
Chaque personnage poss\`ede plusieurs classes, le joueur peux decider de changer la classes de ces personnages pendant un combat, permettant de faire varier les strat\'egies en fonction de l'adversaire (phase de rage, strat\'egie sp�ciale li\'ee a un adversaire, etc...)\\\\
En fonction de la fr\'equence d\textquotesingle utilisation d'une classe dans un combat, celle ci gagne plus ou moins de pourcentage de la somme d\textquotesingle experience donn\'e par ce combat.

\chapter{Planning de r\'ealisation}
Le planning du projet seras g\'erer de fa�on agile (Sprint chaque semaine + r\'eunion tout les jours de 15min pour faire un bilan rapide).\\
Cependant, nous allons r\'ealiser plusieurs it\'eration du projet. Chaque it\'eration auras pour but de rendre un produit fonctionnel, consistant en une amelioration du precedent.
\begin{itemize}
	\item Iteration 1: Realiser un combat entre deux IA basique (Allie (programmable) / Enemie)
	\item Iteration 2: Realiser un systeme basique de classe 
	\item Iteration 3: Realiser un systeme de niveau + inventaire
	\item Iteration 4: Realiser un systeme d'histoire / quete basique
	\item Iteration 5: Ajout d'IA enemie
	\item Iteration 6: Amelioration du systeme d'histoire / quete (branche)
\end{itemize}
A la fin de ces iterations, nous pourrons considerer que la partie obligatoire est finies. Nous pr�voyons au cas ou de r�aliser quelque bonus plus ou moins dur, pouvant �tre r�aliser en plus:
\begin{itemize}
	\item Ajout des achievements
	\item Beaucoup d'item
	\item Plus de classes
	\item Ladder solo
	\item Multijoueur
	\item Ladder multijoueur
	\item Animation entiere du personnage
\end{itemize}

\chapter{Design}

Le jeu sera r\'ealis\'e en pixel art. Chaque partie du corps des personnages et chaque item seront en r�alit� diff\'erents calques que nous superposeront pour cr\'eer les images finales. \\
Cela nous permets de r\'ealiser rapidement un grand nombre de sets de personnages et de rendre nos personnages principaux facilement personnalisables. 
Voici un exemple de nos premi\`eres r\'ealisations:\\
\\
\\
\includegraphics[width=0.3\textwidth]{character-base}
\includegraphics[width=0.3\textwidth]{character-head}
\includegraphics[width=0.3\textwidth]{character-robe}
\includegraphics[width=0.3\textwidth]{character-wand}
\\
\\
\\
\begin{center}
\includegraphics[width=0.3\textwidth]{character}
\end{center}

\chapter{Communication}
La communication se feras via un Website/devblog et sera appui\'ee par un compte twitter qui d\'etaillera l'avancement du jeu en postant des photos, vid\'dos de gameplay etc. Le but \'etant d'attirer le plus de personnes \`a s'int�resser au projet avant m\^eme la sortie du jeu. C'est notamment pour cela qu'une phase de beta test sera disponible aux utilisateurs les plus interess\'ess\'es.\\
 La beta test sera g\'er\'ee a l'aide de la google d�velopper console. Le site proposera \'egalement l'inscription \`a une newsletter. 

\chapter{Mon\'etisation}
La mon\'etisation du jeu se fera via les achats in-app et les pubs. 
Les achats in-app ne devront pas bousculer l'\'equilibre du jeu, notamment dans l'optique d'un multijoueur le plus strat\'egique possible. Nous pr\'evoyons pour le moment l'achat de packs d'xp qui permettrait de lvl up plus vite. \\
Les pubs se voudront non-intrusive pour ne pas g\^acher l'\'experience utilisateur. Elles pourront \^etre par exemple integr\'ees sous forme de banni�re a l'\'ecran de score d'un combat. Elles ne devront pas g\^ener la navigation. Nous utiliserons des api comme admob (utilis\'e par Rovio, Backflip Studio, Fingersoft... )pour int\'egrer des pubs cibl\'ees et donc augmenter nos revenus.

\chapter{Elements \`a v\'erifier en fin de projet}
\begin{itemize}
	\item Un syst\`eme d'IA fonctionnel (voir chapitre Syst\`eme d'IA)
	\item Une interface agr\'eable
	\item Un mode solo fonctionnel
	\item Un multijoueur fonctionnel
	\item La pr\'esence d'achievements
\end{itemize}

\chapter{Bonus}
\begin{itemize}
	\item Une mon\'etisation bien int\'egr\'ee
\end{itemize}

\end{document}
