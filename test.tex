\documentclass[16pt, oneside]{report}
\usepackage[francais]{babel}
\usepackage{textcomp}
 \usepackage[pdftex]{graphicx}
 \usepackage{grffile}  
\font\myfont=cmr12 at 40pt
\graphicspath{{./images/}{./}}  


\title{{\myfont Projet libre}}
\author{mbourlet, gfernand, fbuoro, mchoong \#e1r6}
\date{}

\begin{document}

\maketitle

\chapter{Pr\'esentation du projet}
Le but du projet est de r\'ealiser un jeu vid\'eo natif sur Android.\\
Ce jeu seras un RPG au combat tour par tour mais avec un profond aspect strat\'egique, le joueur ne pourra pas choisir ses actions \`a chaque tour, il devra \'a la place programmer une IA plus ou moins basique (voir plus loin).\\
Le projet se d\'ecomposeras\ en deux parties : l'engine et le jeu

\section{Engine}
L\textquotesingle engine seras un moteur basique pour le jeu, il devras \^etre r\'eutilisable sur diff\'erents projets. Son but sera d\textquotesingle acc\'el\'erer le d\'eveloppement futur et actuel du jeu. Il se divise en plusieurs Manager.

\subsection{Game}
La classe Game est la classe centrale de l'engine. C'est elle qui g\`ere la logique du programme avec son son syst\`eme de state.\\
Au lancement la classe tente de se connecter automatiquement \`a l'api google pour que l'utilisateur se connecte et profite des achievements, de la sauvegarde sur le cloud etc.\\
L'utilisateur peut refuser la connection. Si il refuse la connection 3 fois de suite au lancement, le programme ne lancera plus la connection automatique. Cette donn\'ee est stock\'ee dans un fichier en cache.\\
Avant de lancer le jeu, la classe game va tenter de charger une sauvegarde via la DataManager.

\subsection{Le DataManager}
Le DataManager est celui qui g\`ere les sauvegardes de l'utilisateur. Il offre deux possibilit\'es:
\begin{itemize}
	\item Sauvegarder en local: Le fichier est stock\'e sur le telephone de l'utilisateur dans les fichiers de l'application.
	\item Sauvegarder sur le cloud: Si l'utilisateur se connecte avec son compte Google+ et nous en donne l'autorisation, le DataManager sauvegarde \'egalement sur le cloud.\\
\end{itemize}
 Le travail du DataManager est de faciliter les taches de sauvegarde et de chargement des donn\'ees. Il permet lors du chargement de choisir automatiquement la bonne sauvegarde selon un timestamp ajout\'e au fichier et en g\'erant les conflits si n\'ecessaire. Lors de la sauvegarde il s'occupe de sauvegarder soit en local soit en local + cloud selon l'\'etat de connexion \`a l'API google.


\subsection{State}
Les states sont les diff\'erents �tats du jeu, ce sont elles qui permettront de g\'erer la navigation dans l'application. La game comprends une stack de State.\\
 Imaginons par exemple que l'utilisateur soit dans le menu principal, c'est donc la state li\'ee a ce menu qui recevra toutes les actions de l'utilisateur. Si il lance une partie, la state de la partie se retrouvera en haut de la stack recevant ainsi toutes les instructions de la game, c'est elle qui pourra dessiner et recevoir les inputs de l'utilisateur. \\
 Si l'utilisateur veut revenir a l'\'etat precedent qui est donc le menu principal nous n'avons plus qu'\`a d\'epiler la state de la partie afin que la state du menu reprenne le contr\^ole sur le dessin et les inputs.


\section{Projet}
Le projet en lui m\^eme utiliseras l\textquotesingle engine. Il devras utilis\'e toutes les technologies offerte par Google, tel que le stockage sur le Cloud, le multi-joueur, les hauts-faits etc.

\chapter{Syst\'eme d\textquotesingle IA}
Le joueur pourras contr\^oler une \'equipe de 3 personnages.\\
Chaque personnage pourras avoir de nombreuses classes (Guerrier, Tank, Mage, Sorcier, Archer, Voleur...)\\
Chaque classe de chaque personnage auras son propre niveau.\\
Quand un personnage augmente de niveau dans une classe, il gagne des outils pour programmer son IA li\'e a cette classe, ces outils peuvent \^etre d\'ecomposer sous plusieurs types:
\begin{itemize}
	\item Condition: Permet de conditionner le comportement de l\textquotesingle IA
	\item Action: Cela peut \^etre des sorts, des attaques, etc...
	\item Slot: Permet de poser une nouvelle paire de conditions/actions
\end{itemize}

\section{Classe ?}
Chaque personnage poss\`ede plusieurs classes, le joueur peux decider de changer la classes de ces personnages pendant un combat, permettant de faire varier les strat\'egies en fonction de l'adversaire (phase de rage, strat\'egie sp�ciale li\'ee a un adversaire, etc...)\\\\
En fonction de la fr\'equence d\textquotesingle utilisation d'une classe dans un combat, celle ci gagne plus ou moins de pourcentage de la somme d\textquotesingle experience donn\'e par ce combat.

\chapter{Planning de r\'ealisation}
Le planning du projet seras g\'erer de fa�on agile (Sprint chaque semaine + r\'eunion tout les jours de 15min pour faire un bilan rapide).\\
Cependant, nous allons r\'ealiser plusieurs it\'eration du projet. Chaque it\'eration auras pour but de rendre un produit fonctionnel.
\begin{itemize}
	\item It\'eration 1: R\'ealiser tout le syst�me de classe / level / item + r�alisation d'un premier syst�me de combat contre une IA ennemie basique.
	\item It\'eration 2: R\'ealiser la partie histoire du jeu, en d\'eveloppant les diff�rentes IA enemies ainsi que le syst\`eme d'histoire branch\'ee.
	\item It\'eration 3: R\'ealiser la partie multi-joueur du jeu, avec un syst\`eme de ladder.
\end{itemize}
L'it\'eration 1 auras une dur\'ee de 3 mois.\\
Les it\'eration 2 et 3 auront chacune une dur\'ee de 2 mois sachant qu'elle seront r\'ealiser en parall\`ele par deux \'equipe de deux.

\chapter{Design}

Le jeu sera r\'ealis\'e en pixel art a l'aide du logiciel Pixel Edit. Chaque partie du corps de nos personnages et chaque item seront diff\'erents calques que nous superposeront pour cr\'eer nos personnages. \\
Cela nous permets de r\'ealiser rapidement un grand nombre de sets de personnages et de rendre nos personnages principaux facilement personnalisables. 
Voici un exemple de nos premi\`eres r\'ealisations:\\
\\
\\
\includegraphics[width=0.3\textwidth]{character-base}
\includegraphics[width=0.3\textwidth]{character-head}
\includegraphics[width=0.3\textwidth]{character-robe}
\includegraphics[width=0.3\textwidth]{character-wand}
\\
\\
\\
\begin{center}
\includegraphics[width=0.3\textwidth]{character}
\end{center}

\chapter{Communication}
La communication se feras via un Website/devblog et sera appui\'ee par un compte twitter qui d\'etaillera l'avancement du jeu en postant des photos, vid\'dos de gameplay etc. Le but \'etant d'attirer le plus de personnes \`a s'int�resser au projet avant m\^eme la sortie du jeu. C'est notamment pour cela qu'une phase de beta test sera disponible aux utilisateurs les plus interess\'ess\'es.\\
 La beta test sera g\'er\'ee a l'aide de la google d�velopper console. Le site proposera \'egalement l'inscription \`a une newsletter. 

\chapter{Mon\'etisation}
La mon\'etisation du jeu se fera via les achats in-app et les pubs. 
Les achats in-app ne devront pas bousculer l'\'equilibre du jeu, notamment dans l'optique d'un multijoueur le plus strat\'egique possible. Nous pr\'evoyons pour le moment l'achat de packs d'xp qui permettrait de lvl up plus vite. \\
Les pubs se voudront non-intrusive pour ne pas g\^acher l'\'experience utilisateur. Elles pourront \^etre par exemple integr\'ees sous forme de banni�re a l'\'ecran de score d'un combat. Elles ne devront pas g\^ener la navigation. Nous utiliserons des api comme admob (utilis\'e par Rovio, Backflip Studio, Fingersoft... )pour int\'egrer des pubs cibl\'ees et donc augmenter nos revenus.

\chapter{Elements \`a v\'erifier en fin de projet}
\begin{itemize}
	\item Un syst\`eme d'IA fonctionnel (voir chapitre Syst\`eme d'IA)
	\item Une interface agr\'eable
	\item Un mode solo fonctionnel
	\item Un multijoueur fonctionnel
	\item La pr\'esence d'achievements
\end{itemize}

\chapter{Bonus}
\begin{itemize}
	\item Une mon\'etisation bien int\'egr\'ee
\end{itemize}

\end{document}
